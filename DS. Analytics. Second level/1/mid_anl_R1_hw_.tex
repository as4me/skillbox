% Options for packages loaded elsewhere
\PassOptionsToPackage{unicode}{hyperref}
\PassOptionsToPackage{hyphens}{url}
%
\documentclass[
]{article}
\usepackage{amsmath,amssymb}
\usepackage{lmodern}
\usepackage{ifxetex,ifluatex}
\ifnum 0\ifxetex 1\fi\ifluatex 1\fi=0 % if pdftex
  \usepackage[T1]{fontenc}
  \usepackage[utf8]{inputenc}
  \usepackage{textcomp} % provide euro and other symbols
\else % if luatex or xetex
  \usepackage{unicode-math}
  \defaultfontfeatures{Scale=MatchLowercase}
  \defaultfontfeatures[\rmfamily]{Ligatures=TeX,Scale=1}
\fi
% Use upquote if available, for straight quotes in verbatim environments
\IfFileExists{upquote.sty}{\usepackage{upquote}}{}
\IfFileExists{microtype.sty}{% use microtype if available
  \usepackage[]{microtype}
  \UseMicrotypeSet[protrusion]{basicmath} % disable protrusion for tt fonts
}{}
\makeatletter
\@ifundefined{KOMAClassName}{% if non-KOMA class
  \IfFileExists{parskip.sty}{%
    \usepackage{parskip}
  }{% else
    \setlength{\parindent}{0pt}
    \setlength{\parskip}{6pt plus 2pt minus 1pt}}
}{% if KOMA class
  \KOMAoptions{parskip=half}}
\makeatother
\usepackage{xcolor}
\IfFileExists{xurl.sty}{\usepackage{xurl}}{} % add URL line breaks if available
\IfFileExists{bookmark.sty}{\usepackage{bookmark}}{\usepackage{hyperref}}
\hypersetup{
  pdftitle={Домашнее задание},
  hidelinks,
  pdfcreator={LaTeX via pandoc}}
\urlstyle{same} % disable monospaced font for URLs
\usepackage[margin=1in]{geometry}
\usepackage{color}
\usepackage{fancyvrb}
\newcommand{\VerbBar}{|}
\newcommand{\VERB}{\Verb[commandchars=\\\{\}]}
\DefineVerbatimEnvironment{Highlighting}{Verbatim}{commandchars=\\\{\}}
% Add ',fontsize=\small' for more characters per line
\usepackage{framed}
\definecolor{shadecolor}{RGB}{248,248,248}
\newenvironment{Shaded}{\begin{snugshade}}{\end{snugshade}}
\newcommand{\AlertTok}[1]{\textcolor[rgb]{0.94,0.16,0.16}{#1}}
\newcommand{\AnnotationTok}[1]{\textcolor[rgb]{0.56,0.35,0.01}{\textbf{\textit{#1}}}}
\newcommand{\AttributeTok}[1]{\textcolor[rgb]{0.77,0.63,0.00}{#1}}
\newcommand{\BaseNTok}[1]{\textcolor[rgb]{0.00,0.00,0.81}{#1}}
\newcommand{\BuiltInTok}[1]{#1}
\newcommand{\CharTok}[1]{\textcolor[rgb]{0.31,0.60,0.02}{#1}}
\newcommand{\CommentTok}[1]{\textcolor[rgb]{0.56,0.35,0.01}{\textit{#1}}}
\newcommand{\CommentVarTok}[1]{\textcolor[rgb]{0.56,0.35,0.01}{\textbf{\textit{#1}}}}
\newcommand{\ConstantTok}[1]{\textcolor[rgb]{0.00,0.00,0.00}{#1}}
\newcommand{\ControlFlowTok}[1]{\textcolor[rgb]{0.13,0.29,0.53}{\textbf{#1}}}
\newcommand{\DataTypeTok}[1]{\textcolor[rgb]{0.13,0.29,0.53}{#1}}
\newcommand{\DecValTok}[1]{\textcolor[rgb]{0.00,0.00,0.81}{#1}}
\newcommand{\DocumentationTok}[1]{\textcolor[rgb]{0.56,0.35,0.01}{\textbf{\textit{#1}}}}
\newcommand{\ErrorTok}[1]{\textcolor[rgb]{0.64,0.00,0.00}{\textbf{#1}}}
\newcommand{\ExtensionTok}[1]{#1}
\newcommand{\FloatTok}[1]{\textcolor[rgb]{0.00,0.00,0.81}{#1}}
\newcommand{\FunctionTok}[1]{\textcolor[rgb]{0.00,0.00,0.00}{#1}}
\newcommand{\ImportTok}[1]{#1}
\newcommand{\InformationTok}[1]{\textcolor[rgb]{0.56,0.35,0.01}{\textbf{\textit{#1}}}}
\newcommand{\KeywordTok}[1]{\textcolor[rgb]{0.13,0.29,0.53}{\textbf{#1}}}
\newcommand{\NormalTok}[1]{#1}
\newcommand{\OperatorTok}[1]{\textcolor[rgb]{0.81,0.36,0.00}{\textbf{#1}}}
\newcommand{\OtherTok}[1]{\textcolor[rgb]{0.56,0.35,0.01}{#1}}
\newcommand{\PreprocessorTok}[1]{\textcolor[rgb]{0.56,0.35,0.01}{\textit{#1}}}
\newcommand{\RegionMarkerTok}[1]{#1}
\newcommand{\SpecialCharTok}[1]{\textcolor[rgb]{0.00,0.00,0.00}{#1}}
\newcommand{\SpecialStringTok}[1]{\textcolor[rgb]{0.31,0.60,0.02}{#1}}
\newcommand{\StringTok}[1]{\textcolor[rgb]{0.31,0.60,0.02}{#1}}
\newcommand{\VariableTok}[1]{\textcolor[rgb]{0.00,0.00,0.00}{#1}}
\newcommand{\VerbatimStringTok}[1]{\textcolor[rgb]{0.31,0.60,0.02}{#1}}
\newcommand{\WarningTok}[1]{\textcolor[rgb]{0.56,0.35,0.01}{\textbf{\textit{#1}}}}
\usepackage{graphicx}
\makeatletter
\def\maxwidth{\ifdim\Gin@nat@width>\linewidth\linewidth\else\Gin@nat@width\fi}
\def\maxheight{\ifdim\Gin@nat@height>\textheight\textheight\else\Gin@nat@height\fi}
\makeatother
% Scale images if necessary, so that they will not overflow the page
% margins by default, and it is still possible to overwrite the defaults
% using explicit options in \includegraphics[width, height, ...]{}
\setkeys{Gin}{width=\maxwidth,height=\maxheight,keepaspectratio}
% Set default figure placement to htbp
\makeatletter
\def\fps@figure{htbp}
\makeatother
\setlength{\emergencystretch}{3em} % prevent overfull lines
\providecommand{\tightlist}{%
  \setlength{\itemsep}{0pt}\setlength{\parskip}{0pt}}
\setcounter{secnumdepth}{-\maxdimen} % remove section numbering
\usepackage[russian]{babel}
\usepackage{hyperref}
\hypersetup{ colorlinks = true, urlcolor = blue}
\ifluatex
  \usepackage{selnolig}  % disable illegal ligatures
\fi

\title{Домашнее задание}
\author{}
\date{\vspace{-2.5em}}

\begin{document}
\maketitle

\hypertarget{ux437ux430ux434ux430ux43dux438ux435-1}{%
\subsection{Задание 1}\label{ux437ux430ux434ux430ux43dux438ux435-1}}

Вычислите в R:

\begin{itemize}
\tightlist
\item
  \(67 ^ 3 - 112 ^ 2\);
\item
  \(\log(125)\);
\item
  \(\log_3(81)\).
\end{itemize}

\begin{Shaded}
\begin{Highlighting}[]
\DecValTok{67} \SpecialCharTok{**} \DecValTok{3} \SpecialCharTok{{-}} \DecValTok{112} \SpecialCharTok{**} \DecValTok{2}
\end{Highlighting}
\end{Shaded}

\begin{verbatim}
## [1] 288219
\end{verbatim}

\begin{Shaded}
\begin{Highlighting}[]
\FunctionTok{log}\NormalTok{(}\DecValTok{125}\NormalTok{)}
\end{Highlighting}
\end{Shaded}

\begin{verbatim}
## [1] 4.828314
\end{verbatim}

\begin{Shaded}
\begin{Highlighting}[]
\FunctionTok{log}\NormalTok{(}\DecValTok{81}\NormalTok{,}\DecValTok{3}\NormalTok{)}
\end{Highlighting}
\end{Shaded}

\begin{verbatim}
## [1] 4
\end{verbatim}

\hypertarget{ux437ux430ux434ux430ux43dux438ux435-2}{%
\subsection{Задание 2}\label{ux437ux430ux434ux430ux43dux438ux435-2}}

В векторе \texttt{flights\_d} сохранено число вылетов из аэропорта А, а
в векторе \texttt{flights\_a} --- число прилетов в этот аэропорт за
неделю.

\begin{Shaded}
\begin{Highlighting}[]
\NormalTok{flights\_d }\OtherTok{\textless{}{-}} \FunctionTok{c}\NormalTok{(}\DecValTok{140}\NormalTok{, }\DecValTok{150}\NormalTok{, }\DecValTok{100}\NormalTok{, }\DecValTok{90}\NormalTok{, }\DecValTok{230}\NormalTok{, }\DecValTok{240}\NormalTok{, }\DecValTok{165}\NormalTok{)}
\NormalTok{flights\_a }\OtherTok{\textless{}{-}} \FunctionTok{c}\NormalTok{(}\DecValTok{65}\NormalTok{, }\DecValTok{145}\NormalTok{, }\DecValTok{80}\NormalTok{, }\DecValTok{87}\NormalTok{, }\DecValTok{220}\NormalTok{, }\DecValTok{268}\NormalTok{, }\DecValTok{216}\NormalTok{)}
\end{Highlighting}
\end{Shaded}

2.1. Сколько вылетов из аэропорта А было зафиксировано в среду?

2.2. На сколько число вылетов во вторник больше числа прилетов во
вторник?

2.3. Во сколько раз число вылетов в воскресенье больше числа прилетов в
воскресенье?

2.4. Сколько всего вылетов из аэропорта А было зафиксировано за неделю?

\hypertarget{ux437ux430ux434ux430ux43dux438ux435-3}{%
\subsection{Задание 3}\label{ux437ux430ux434ux430ux43dux438ux435-3}}

В векторе \texttt{cats} сохранены значения весов кошек в килограммах:

\begin{Shaded}
\begin{Highlighting}[]
\NormalTok{cats }\OtherTok{\textless{}{-}} \FunctionTok{c}\NormalTok{(}\FloatTok{4.765}\NormalTok{, }\FloatTok{3.230}\NormalTok{, }\FloatTok{1.256}\NormalTok{, }\FloatTok{1.780}\NormalTok{, }\FloatTok{2.583}\NormalTok{, }\FloatTok{2.781}\NormalTok{, }\FloatTok{3.945}\NormalTok{, }\FloatTok{2.345}\NormalTok{)}
\end{Highlighting}
\end{Shaded}

3.1. Используя R, выведите ответы на вопросы.

\begin{itemize}
\tightlist
\item
  Сколько всего кошек было взвешено?
\item
  Какой вес был у самой тяжелой кошки? А у легкой?
\end{itemize}

3.2. Создайте вектор \texttt{cats.round} со значениями весов кошек в
килограммах, округленных в меньшую сторону.

3.3. Создайте вектор \texttt{cats\_g} со значением весов кошек в
граммах.

\hypertarget{ux437ux430ux434ux430ux43dux438ux435-4}{%
\subsection{Задание 4}\label{ux437ux430ux434ux430ux43dux438ux435-4}}

В векторе \texttt{nums} сохранены строки, в которых записаны дробные
числа с неправильным десятичным разделителем. Создайте числовой вектор
\texttt{correct} с корректными значениями --- правильными дробными
числами.

\begin{Shaded}
\begin{Highlighting}[]
\NormalTok{nums }\OtherTok{\textless{}{-}} \FunctionTok{c}\NormalTok{(}\StringTok{"2,6"}\NormalTok{, }\StringTok{"2,71"}\NormalTok{, }\StringTok{"3,5"}\NormalTok{, }\StringTok{"4,8"}\NormalTok{, }\StringTok{"8,9"}\NormalTok{, }\StringTok{"9,21"}\NormalTok{)}
\end{Highlighting}
\end{Shaded}

\hypertarget{ux437ux430ux434ux430ux43dux438ux435-5}{%
\subsection{Задание 5}\label{ux437ux430ux434ux430ux43dux438ux435-5}}

Известно, что в таблице хранятся показатели по 3 странам за 5 лет.
Таблица выглядит примерно так:

\begin{table}[]
\centering
\begin{tabular}{|l|l|}
\hline
\textbf{country} & \textbf{year} \\ \hline
France           & 2000          \\ \hline
France           & 2001          \\ \hline
France           & 2002          \\ \hline
France           & 2003          \\ \hline
France           & 2004          \\ \hline
Italy            & 2000          \\ \hline
Italy            & 2001          \\ \hline
Italy            & 2002          \\ \hline
Italy            & 2003          \\ \hline
Italy            & 2004          \\ \hline
Spain            & 2000          \\ \hline
Spain            & 2001          \\ \hline
Spain            & 2002          \\ \hline
Spain            & 2003          \\ \hline
Spain            & 2004          \\ \hline
\end{tabular}
\end{table}

\begin{itemize}
\item
  Создайте вектор \texttt{country} с названиями стран, то есть вектор,
  который послужил бы первым столбцом таблицы выше.
\item
  Создайте вектор \texttt{year} с годами, который мог бы послужить
  вторым столбцом таблицы выше.
\item
  Объедините полученные векторы в датафрейм \texttt{dat} таким образом,
  чтобы в итоге получилась таблица, представленная выше.
\end{itemize}

\hypertarget{ux437ux430ux434ux430ux43dux438ux435-6}{%
\subsection{Задание 6}\label{ux437ux430ux434ux430ux43dux438ux435-6}}

Даны векторы \texttt{v} и \texttt{w}:

\begin{Shaded}
\begin{Highlighting}[]
\NormalTok{v }\OtherTok{\textless{}{-}} \FunctionTok{c}\NormalTok{(}\DecValTok{2}\NormalTok{, }\DecValTok{8}\NormalTok{, }\DecValTok{9}\NormalTok{, }\DecValTok{11}\NormalTok{, }\DecValTok{13}\NormalTok{, }\DecValTok{0}\NormalTok{)}
\NormalTok{w }\OtherTok{\textless{}{-}} \FunctionTok{c}\NormalTok{(}\DecValTok{0}\NormalTok{, }\DecValTok{3}\NormalTok{, }\DecValTok{7}\NormalTok{, }\DecValTok{0}\NormalTok{, }\DecValTok{5}\NormalTok{, }\DecValTok{1}\NormalTok{)}
\end{Highlighting}
\end{Shaded}

6.1. Используя векторы \texttt{v} и \texttt{w}, создайте:

\begin{itemize}
\item
  Матрицу A, где векторы \texttt{v} и \texttt{w} являются строками
  матрицы.
\item
  Матрицу B, где векторы \texttt{v} и \texttt{w} являются столбцами
  матрицы.
\end{itemize}

6.2. Выведите на экран элемент матрицы B в 2 столбце и 3 строке.

6.3. Замените элемент в 4 строке второго столбца B на 0.

6.4. Используя R, ответьте на следующий вопрос. В какой матрице сумма
элементов второй строки больше?

\end{document}
